% Preamble
\documentclass[11pt]{PyRollDocs}
\usepackage{textcomp}

\addbibresource{refs.bib}

% Document
\begin{document}

    \title{The Wusatowski Spreading PyRoll Plugin}
    \author{Max Weiner, Christoph Renzing}
    \date{\today}

    \maketitle

    This plugin provides a spreading modelling approach with Wusatowski's formula for flat rolling.


    \section{Model approach}\label{sec:model-approach}

    \subsection{Wusatowski's spread equation}\label{subsec:wusatowski's-spread-equation}

    \textcite{Wusatowski1969} proposed \autoref{eq:wusatowski} for estimation of spreading in flat rolling,
    where $\gamma = \frac{h_1}{h_0}$ is the compression. $h$ and $b$ are height and width of the workpiece with the indices
    0 and 1 denoting the incoming respectively the outgoing profile. $a$, $c$, $d$ and $f$ are correction
    coefficients for temperature, velocity, material and friction, respectively.

    \begin{equation}
        \beta = \frac{b_1}{b_0} = a \times c \times d \times f \times \gamma^{-w}
        \label{eq:wusatowski}
    \end{equation}

    \noindent The velocity coefficient $c$ can be assumed as below in dependence on the velocity $v$.

    \begin{equation}
        c = \left(-0.002958 + 0.00341 \gamma \right) v + 1.07168 - 0.10431 \gamma
        \label{eq:velocity-coefficient}
    \end{equation}

    $w$ is the spread exponent.
    As for strains $\epsilon$ smaller than 50\%, \textcite{Wusatowski1969} proposed ~\autoref{eq:exponent_low}.
    For strains higher than 50\% \textcite{Wusatowski1969} used a adjusted version of the exponent $w_{high}$.
    The origin of this equation was first given by \textcite{Hill1955} which derived the equation from plastic stress-strain equations from \textcite{Mises1913}.
    In the following expressions, $R$ is the roll radius.

    \begin{subequations}
        \begin{equation}
            w_{low} = 10^{ \num{-1.269} \left( \frac{h_0}{2 R} \right)^{\num{0.556}} \frac{b_0}{h_0} }
            \label{eq:exponent_low}\\
        \end{equation}
        \begin{equation}
            w_{high} = 10^{ \num{-3.457} \left( \frac{h_0}{2 R} \right)^{\num{0.958}} \frac{b_0}{h_0} }
            \label{eq:exponent_high}\\
        \end{equation}
    \end{subequations}


    \section{Usage instructions}\label{sec:usage-instructions}

    The plugin can be loaded under the name \texttt{pyroll\_wusatowski\_spreading}.

    An implementation of the \lstinline{spread} hook on \lstinline{RollPass} is provided,
    calculating the spread using the equivalent rectangle approach and Wusatowski's model.

    Several additional hooks on \lstinline{RollPass} are defined, which are used in spread calculation, as listed in \autoref{tab:hookspecs}.
    Base implementations of them are provided, so it should work out of the box.
    For \lstinline{wusatowski_exponent_low}, \lstinline{wusatowski_exponent_high} and \lstinline{wusatowski_velocity_coefficient}
    the equations~\ref{eq:exponent_low},~\ref{eq:exponent_high} and~\ref{eq:velocity-coefficient} are implemented.
    The others default to \num{1}.
    Provide your own hook implementations or set attributes on the \lstinline{RollPass} instances to alter the spreading behavior.

    \begin{table}
        \centering
        \caption{Hooks specified by this plugin. Symbols as in \autoref{eq:wusatowski}.}
        \label{tab:hookspecs}
        \begin{tabular}{ll}
            \toprule
            Hook name                                     & Meaning                                \\
            \midrule
            \texttt{wusatowski\_temperature\_coefficient} & temperature correction coefficient $a$ \\
            \texttt{wusatowski\_velocity\_coefficient}    & velocity correction coefficient $c$    \\
            \texttt{wusatowski\_material\_coefficient}    & material correction coefficient $d$    \\
            \texttt{wusatowski\_friction\_coefficient}    & friction correction coefficient $f$    \\
            \texttt{wusatowski\_exponent\_low}            & spread exponent $w_{low}$              \\
            \texttt{wusatowski\_exponent\_high}           & spread exponent $w_{high}$             \\
            \bottomrule
        \end{tabular}
    \end{table}

    \printbibliography

\end{document}